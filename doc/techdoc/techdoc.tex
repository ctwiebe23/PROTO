%%%%%%%%%%%%%%%%%%%%%%%%%%%%%%%%%%%%%%%%%%%%%%%%%%%%%%%%%%%%%%%%%%%%%%%%%%%%%%%
%
% Carston Wiebe
% MAKE Design Document
%
%%%%%%%%%%%%%%%%%%%%%%%%%%%%%%%%%%%%%%%%%%%%%%%%%%%%%%%%%%%%%%%%%%%%%%%%%%%%%%%

\documentclass[12pt]{scrartcl} % or scrbook

\usepackage{xcolor}
\usepackage[
  colorlinks=true,
  urlcolor=darkblue,
  citecolor=darkergreen,
  linkcolor=darkblue,
  plainpages=false,
  pdfpagelabels
]{hyperref}
\usepackage{graphicx}
\graphicspath{ {./images} }
\usepackage{minted}
\usepackage{fancyhdr}

% Colors
\definecolor{darkred}{rgb}{0.75,0,0}
\definecolor{darkblue}{rgb}{0,0,0.5}
\definecolor{darkgreen}{rgb}{0,0.5,0}
\definecolor{darkergreen}{rgb}{0,0.75,0}
\definecolor{darkmagenta}{rgb}{0.55,0,0.55}
\definecolor{lightgrey}{gray}{0.9}
\definecolor{left}{HTML}{041832}
\definecolor{secondary}{HTML}{241024}

% Formatting
\setlength{\parindent}{0pt} \setlength{\parskip}{.25cm} \pagestyle{fancy}

\title{MAKE}
\subtitle{
  Python Toolbox for \\
  Robotics Education
}
\author{
  Carston Wiebe \\
  \href{mailto:cwiebe3@huskers.unl.edu}{cwiebe3@huskers.unl.edu} \\
}

\date{
  2024-07 \\
  Version 4.0
}

\begin{document}

%%%%%%%%%%%%%%%%%%%%%%%%%%%%%%%%%%%%%%%%%%%%%%%%%%%%%%%%%%%%%%%%%%%%%%%%%%%%%%%
%%%%%%%%%%%%%%%%%%%%%%%%%%%%%%%%%%%%%%%%%%%%%%%%%%%%%%%%%%%%%%%%%%%%%%%%%%%%%%%

\maketitle
\thispagestyle{empty}

\vfill

\begin{abstract}
  Make is a toolbox of functions and classes that mostly serves as a wrapper
  for CIRCUITPython and is used by the University of Nebraska Lincoln's PROTO
  RSO to aid in robotics education. The goal of the project is to create a
  psuedo-programming language that middle school/upper elementary aged students
  can use to code robots and learn simple programming skills.
\end{abstract}

%%%%%%%%%%%%%%%%%%%%%%%%%%%%%%%%%%%%%%%%%%%%%%%%%%%%%%%%%%%%%%%%%%%%%%%%%%%%%%%
\newpage
\clearpage
%%%%%%%%%%%%%%%%%%%%%%%%%%%%%%%%%%%%%%%%%%%%%%%%%%%%%%%%%%%%%%%%%%%%%%%%%%%%%%%

\section*{REVISIONS}

\begin{table}[ht]
  \centering
  \def\arraystretch{1.5}
  \begin{tabular}{|l|l|l|}
    \hline
    DATE       & CHANGE(S)                    & AUTHOR(S) \\
    \hline
    2024-07-19 & Proofreading                 & C. Wiebe  \\
    \hline
    2024-04-29 & Translated doc to LaTeX      & C. Wiebe  \\
    \hline
    2024-04-21 & large\_motor \& small\_motor & C. Wiebe  \\
    \hline
    2024-04-12 & button \& functions          & C. Wiebe  \\
    \hline
    2024-04-10 & PHILOSOPHY AND STANDARDS     & C. Wiebe  \\
    \hline
    2024-04-06 & Techdoc creation             & C. Wiebe  \\
    \hline
  \end{tabular}
\end{table}

%%%%%%%%%%%%%%%%%%%%%%%%%%%%%%%%%%%%%%%%%%%%%%%%%%%%%%%%%%%%%%%%%%%%%%%%%%%%%%%
\newpage
%%%%%%%%%%%%%%%%%%%%%%%%%%%%%%%%%%%%%%%%%%%%%%%%%%%%%%%%%%%%%%%%%%%%%%%%%%%%%%%

\tableofcontents

%%%%%%%%%%%%%%%%%%%%%%%%%%%%%%%%%%%%%%%%%%%%%%%%%%%%%%%%%%%%%%%%%%%%%%%%%%%%%%%
\newpage
%%%%%%%%%%%%%%%%%%%%%%%%%%%%%%%%%%%%%%%%%%%%%%%%%%%%%%%%%%%%%%%%%%%%%%%%%%%%%%%

\section{INTRODUCTION}

\subsection{PURPOSE}

Created for the University of Nebraska-Lincoln's PROTO student organization
with the goal of providing a psuedo programming language that can be used by
middle and elementary school aged students.

\subsection{DESCRIPTION}

Wrapper for CIRCUITPython (also written in Python) to be used on the MakerPI
RP2040 and PROTO's componants. Branches out into creating unique functions and
classes, but mostly serves to simplify existing code in both the standard
Python toolset and in CIRCUITPython.

\subsection{PHILOSOPHY AND STANDARDS}

A minimum amount of knowledge on coding should be assumed and expected, and the
users may not have access to a proper IDE with features like intelligent syntax
highlighting or auto-complete. As such, there are two standards to use
depending on whether code will be outward facing-- to be used by students-- or
inward facing.

For outward ('public') classes, functions, and variables:

\begin{itemize}
  \item[-] Priotitize short, easy to spell names-- ideally one word
  \item[-] Limit the number of function/constructor arguments
  \item[-] Make lines of code read like English; i.e
    \mintinline{python}{until(button.pressed)}
  \item[-] Remove complexities whenever possible
\end{itemize}

For inward classes, functions, and variables you can follow standard Python
form for the most part:

\begin{itemize}
  \item[-] \mintinline{python}{lower_snake_case} (even for classes)
  \item[-] Each class gets its own file
  \item[-] Limit lines to 80 characters when possible
  \item[-] Use type hints for both function arguments and return types
  \item[-] For functions and variables that are not meant to public, prefix
    them with \mintinline{python}{__} to prevent them from being interfered
    with.
\end{itemize}

Lastly, there are some rules required for compatability with CIRCUITPython and
BAPCAT (described later):

\begin{itemize}
  \item[-] Not all Python libraries are available on the PIs; always check to
    make sure your code still runs after adding an import. We might have to
    design implentations of some things ourselves.
\end{itemize}

%%%%%%%%%%%%%%%%%%%%%%%%%%%%%%%%%%%%%%%%%%%%%%%%%%%%%%%%%%%%%%%%%%%%%%%%%%%%%%%
\newpage
%%%%%%%%%%%%%%%%%%%%%%%%%%%%%%%%%%%%%%%%%%%%%%%%%%%%%%%%%%%%%%%%%%%%%%%%%%%%%%%

\section{BAPCAT}

%%%%%%%%%%%%%%%%%%%%%%%%%%%%%%%%%%%%%%%%%%%%%%%%%%%%%%%%%%%%%%%%%%%%%%%%%%%%%%%
\newpage
%%%%%%%%%%%%%%%%%%%%%%%%%%%%%%%%%%%%%%%%%%%%%%%%%%%%%%%%%%%%%%%%%%%%%%%%%%%%%%%

\section{CONTENT}

\subsection{CLASSES}

\subsubsection{button}

\begin{minted}[frame=lines,
               framesep=2mm,
               baselinestretch=1.2,
               fontsize=\footnotesize]{python}
  button(pin)
\end{minted}

Represents a button, either one of the two mounted to the board or one attached
later. Requires only a port to be constructed, and can be built from either one
of the \mintinline{python}{BUTTON_PIN}s or one of the
\mintinline{python}{GROVE_PINs}.

\begin{minted}[frame=lines,
               framesep=2mm,
               baselinestretch=1.2,
               fontsize=\footnotesize]{python}
  button.pressed()
\end{minted}

Returns \mintinline{python}{true} if the button is pressed, and
\mintinline{python}{false} otherwise.

\subsubsection{distance\_sensor}

\subsubsection{large\_motor}

\begin{minted}[frame=lines,
               framesep=2mm,
               baselinestretch=1.2,
               fontsize=\footnotesize]{python}
  large_motor(pin_set, direction = 1)
\end{minted}

Represents a DC motor mounted on one of the 2 DC motor ports. Requires only a
\mintinline{python}{pin_set} in \mintinline{python}{DC_PIN}s to be constructed,
which limits the number of \mintinline{python}{large_motor}s to 2.

\mintinline{python}{direction} is an optional variable that changes the
direction of the motor, either positive or negative.

\begin{minted}[frame=lines,
               framesep=2mm,
               baselinestretch=1.2,
               fontsize=\footnotesize]{python}
  large_motor.spin(speed, time = None)
\end{minted}

Takes a \mintinline{python}{speed} in the range $[-100, 100]$ (all values
outside the range are constrained) and runs the motor at that speed.
Optionally, a \mintinline{python}{time} can be passed in seconds and the motor
will only spin for the allotted time before stopping.

\begin{minted}[frame=lines,
               framesep=2mm,
               baselinestretch=1.2,
               fontsize=\footnotesize]{python}
  large_motor.stop()
\end{minted}

Equivalent to \mintinline{python}{large_motor.spin(0)}.

\subsubsection{small\_motor}

\begin{minted}[frame=lines,
               framesep=2mm,
               baselinestretch=1.2,
               fontsize=\footnotesize]{python}
  small_motor(pin_set, direction = 1)
\end{minted}

Represents a servo motor mounted on one of the 4 servo motor ports or one of
the grove ports. Requires only a \mintinline{python}{pin_set} to be
constructed.

\mintinline{python}{direction} is an optional variable that changes the
direction of the motor, either positive or negative.

\begin{minted}[frame=lines,
               framesep=2mm,
               baselinestretch=1.2,
               fontsize=\footnotesize]{python}
  small_motor.spin(speed, time = None)
\end{minted}

Takes a \mintinline{python}{speed} in the range $[-100, 100]$ (all values
outside the range are constrained) and runs the motor at that speed.
Optionally, a \mintinline{python}{time} can be passed in seconds and the motor
will only spin for the allotted time before stopping.

\begin{minted}[frame=lines,
               framesep=2mm,
               baselinestretch=1.2,
               fontsize=\footnotesize]{python}
  small_motor.stop()
\end{minted}

Equivalent to \mintinline{python}{small_motor.spin(0)}.

\subsubsection{wagon}

\subsection{FUNCTIONS}

\subsubsection{pause}

\begin{minted}[frame=lines,
               framesep=2mm,
               baselinestretch=1.2,
               fontsize=\footnotesize]{python}
  pause(time = None)
\end{minted}

If a \mintinline{python}{time} is passed, waits for the allotted
\mintinline{python}{time}. Otherwise, waits for $0.005$ seconds.

\subsubsection{until}

\begin{minted}[frame=lines,
               framesep=2mm,
               baselinestretch=1.2,
               fontsize=\footnotesize]{python}
  until(condition)
\end{minted}

Pauses the program until the passed \mintinline{python}{condition} is
satisfied.

\subsection{CONSTANTS}

\subsubsection{BUTTON\_PIN}

\subsubsection{DC\_PIN}

\subsubsection{FRQ}

\subsubsection{GROVE\_PIN}

\subsubsection{reversed}

\subsubsection{SERVO\_PIN}

%%%%%%%%%%%%%%%%%%%%%%%%%%%%%%%%%%%%%%%%%%%%%%%%%%%%%%%%%%%%%%%%%%%%%%%%%%%%%%%
%%%%%%%%%%%%%%%%%%%%%%%%%%%%%%%%%%%%%%%%%%%%%%%%%%%%%%%%%%%%%%%%%%%%%%%%%%%%%%%

\end{document}
