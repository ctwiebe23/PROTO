%%%%%%%%%%%%%%%%%%%%%%%%%%%%%%%%%%%%%%%%%%%%%%%%%%%%%%%%%%%%%%%%%%%%%%%%%%%%%%%
%
% Carston Wiebe
% MAKE Design Document
%
%%%%%%%%%%%%%%%%%%%%%%%%%%%%%%%%%%%%%%%%%%%%%%%%%%%%%%%%%%%%%%%%%%%%%%%%%%%%%%%

\documentclass[12pt]{scrartcl} % or scrbook

\usepackage{xcolor}
\usepackage[
  colorlinks=true,
  urlcolor=darkblue,
  citecolor=darkergreen,
  linkcolor=darkblue,
  plainpages=false,
  pdfpagelabels
]{hyperref}
\usepackage{graphicx}
\graphicspath{ {./images} }
% \usepackage{minted}
\usepackage{fancyhdr}

% Colors
\definecolor{darkred}{rgb}{0.75,0,0}
\definecolor{darkblue}{rgb}{0,0,0.5}
\definecolor{darkgreen}{rgb}{0,0.5,0}
\definecolor{darkergreen}{rgb}{0,0.75,0}
\definecolor{darkmagenta}{rgb}{0.55,0,0.55}
\definecolor{left}{HTML}{041832}
\definecolor{secondary}{HTML}{241024}

\setlength{\parindent}{0pt} \setlength{\parskip}{.25cm} \pagestyle{fancy}

\title{MAKE}
\subtitle{
  Python Toolbox for \\
  Robotics Education
}
\author{
  Carston Wiebe \\
  \href{mailto:cwiebe3@huskers.unl.edu}{cwiebe3@huskers.unl.edu} \\
}

\date{
  2024-04 \\
  Version 3.0
}

\begin{document}

%%%%%%%%%%%%%%%%%%%%%%%%%%%%%%%%%%%%%%%%%%%%%%%%%%%%%%%%%%%%%%%%%%%%%%%%%%%%%%%
%%%%%%%%%%%%%%%%%%%%%%%%%%%%%%%%%%%%%%%%%%%%%%%%%%%%%%%%%%%%%%%%%%%%%%%%%%%%%%%

\maketitle
\thispagestyle{empty}

\vfill

\begin{abstract}
  Abstract TODO
\end{abstract}

%%%%%%%%%%%%%%%%%%%%%%%%%%%%%%%%%%%%%%%%%%%%%%%%%%%%%%%%%%%%%%%%%%%%%%%%%%%%%%%
\newpage
\clearpage
%%%%%%%%%%%%%%%%%%%%%%%%%%%%%%%%%%%%%%%%%%%%%%%%%%%%%%%%%%%%%%%%%%%%%%%%%%%%%%%

\section*{REVISIONS}

\begin{table}[htp]
  \centering
  \def\arraystretch{1.5}
  \begin{tabular}{|l|l|l|}
    \hline
    DATE       & CHANGE(S)                   & AUTHOR(S) \\
    \hline
    2024-04-29 & translated doc to latex     & C. Wiebe  \\
    \hline
    2024-04-21 & largemotor \& button        & C. Wiebe  \\
    \hline
    2024-04-12 & dc\_motors, servos, stopall & C. Wiebe  \\
    \hline
    2024-04-10 & PHILOSOPHY                  & C. Wiebe  \\
    \hline
    2024-04-06 & techdoc creation            & C. Wiebe  \\
    \hline
  \end{tabular}
\end{table}

%%%%%%%%%%%%%%%%%%%%%%%%%%%%%%%%%%%%%%%%%%%%%%%%%%%%%%%%%%%%%%%%%%%%%%%%%%%%%%%
\newpage
%%%%%%%%%%%%%%%%%%%%%%%%%%%%%%%%%%%%%%%%%%%%%%%%%%%%%%%%%%%%%%%%%%%%%%%%%%%%%%%

\tableofcontents

%%%%%%%%%%%%%%%%%%%%%%%%%%%%%%%%%%%%%%%%%%%%%%%%%%%%%%%%%%%%%%%%%%%%%%%%%%%%%%%
\newpage
%%%%%%%%%%%%%%%%%%%%%%%%%%%%%%%%%%%%%%%%%%%%%%%%%%%%%%%%%%%%%%%%%%%%%%%%%%%%%%%

\section{INTRODUCTION}

\subsection{PURPOSE}

Created for the University of Nebraska-Lincoln's SPARK student organization

\subsection{DESCRIPTION}

Wrapper for CIRCUITPython (also written in Python) to be used on the MakerPI
RP2040 in order to allow the simple creation of code by elementary and middle
school students for educational purposes.

\subsection{PHILOSOPHY}

Classes and functions should be written with ease of use in mind-- a minimum
amount of knowledge on coding should be assumed and expected, and the users may
not have access to a proper IDE with features like intelligent syntax
highlighting.

As such, convention should be disregarded when it conflicts with ease of
understanding; for instance, \emph{all} names should be lowercase only so as to
prevent simple issues with capitalization that young users might encounter (and
that won't be caught by the computer without an IDE).

(Incomplete) list of rules:

\begin{itemize}
  \item[-] all flatcase (lowercase, no underscores)
  \item[-] prioritize one-word names
  \item[-] if possible, lines should read like English; i.e. until( button.pressed )
  \item[-] use simple words without complicated spelling
  \item[-] abstract complexities away; i.e. rather than require users to input
    board.GP10, create a dict with the key 10 and value board.GP10
\end{itemize}

%%%%%%%%%%%%%%%%%%%%%%%%%%%%%%%%%%%%%%%%%%%%%%%%%%%%%%%%%%%%%%%%%%%%%%%%%%%%%%%
\newpage
%%%%%%%%%%%%%%%%%%%%%%%%%%%%%%%%%%%%%%%%%%%%%%%%%%%%%%%%%%%%%%%%%%%%%%%%%%%%%%%

\section{CONTENT}
\label{section:detailedDesignDescription}

TODO content

%%%%%%%%%%%%%%%%%%%%%%%%%%%%%%%%%%%%%%%%%%%%%%%%%%%%%%%%%%%%%%%%%%%%%%%%%%%%%%%
%%%%%%%%%%%%%%%%%%%%%%%%%%%%%%%%%%%%%%%%%%%%%%%%%%%%%%%%%%%%%%%%%%%%%%%%%%%%%%%

\end{document}
