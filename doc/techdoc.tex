%%%%%%%%%%%%%%%%%%%%%%%%%%%%%%%%%%%%%%%%%%%%%%%%%%%%%%%%%%%%%%%%%%%%%%%%%%%%%%%
%
% MAKE
%
% AUTHOR(S)
%   Carston Wiebe (cwiebe3@huskers.unl.edu)
%
%%%%%%%%%%%%%%%%%%%%%%%%%%%%%%%%%%%%%%%%%%%%%%%%%%%%%%%%%%%%%%%%%%%%%%%%%%%%%%%

\documentclass[leqno]{article}

% Math
\usepackage{amsmath, amssymb, amsthm, mathtools, blkarray}
% Visual
\usepackage[table]{xcolor}
\usepackage{graphicx, tikz, etoolbox, fancyhdr}
\graphicspath{ {./images} }
% Algorithm
\usepackage[linesnumbered,ruled,vlined]{algorithm2e}

% Formatting
\AtBeginEnvironment{align}{\setcounter{equation}{0}}
\AtBeginEnvironment{alignat}{\setcounter{equation}{0}}
\setlength{\parindent}{0pt}
\setlength{\parskip}{.25cm}
\pagestyle{fancy}

\title{MAKE}
\author{Carston Wiebe}

\begin{document}

\maketitle

\section{REVISIONS}

\begin{table*}[htp]
  \centering
  \def\arraystretch{1.5}
  \begin{tabular}{|l|l|l|}
    \hline
    DATE & CHANGE & AUTHOR(S) \\
    \hline\hline
    2024-04-22 & translated doc to markdown & Carston Wiebe \\\hline
    2024-04-21 & `largemotor` \& `button` & Carston Wiebe \\\hline
    2024-04-12 & `dc\_motors`, `servos`, `stopall` & Carston Wiebe \\\hline
    2024-04-10 & PHILOSOPHY & Carston Wiebe \\\hline
    2024-04-06 & techdoc creation & Carston Wiebe \\\hline
  \end{tabular}
\end{table*}

\section{PURPOSE}

Created for the University of Nebraska-Lincoln's SPARK student organization

\section{DESCRIPTION}

Wrapper for CIRCUITPython (also written in Python) to be used on the MakerPI
RP2040 in order to allow the simple creation of code by elementary and middle
school students for educational purposes.

\section{PHILOSOPHY}

Classes and functions should be written with ease of use in mind-- a minimum
amount of knowledge on coding should be assumed and expected, and the users may
not have access to a proper IDE with features like intelligent syntax
highlighting.

As such, convention should be disregarded when it conflicts with ease of
understanding; for instance, *all* names should be lowercase only so as to
prevent simple issues with capitalization that young users might encounter (and
that won't be caught by the computer without an IDE).

(Incomplete) list of rules:

\begin{itemize}
  \item[-] all flatcase (lowercase, no underscores)
  \item[-] prioritize one-word names
  \item[-] if possible, lines should read like English; i.e.
    `until( button.pressed )`
  \item[-] use simple words without complicated spelling
  \item[-] abstract complexities away; i.e. rather than require users to input
    `board.GP10`, create a dict with the key `10` and value `board.GP10`
\end{itemize}

\section{CONTENT}

\subsection{CLASSES}

\subsubsection{button}

    button( pin )

Represents a button, either one of the two mounted to the board or one attached
later. Requires only a port to be constructed, and can be built from either one
of the `BUTTON\_PIN`s or one of the `GROVE\_PINs`.

pressed

    button.pressed()

Returns `true` if the button is pressed, and `false` otherwise.

\subsubsection{largemotor}

    largemotor( pinset )

Represents a DC motor mounted on one of the 2 DC motor ports. Requires only a
`pinset` in `DC\_PIN`s to be constructed, which limits the number of
`largemotor`s to 2.

spin

    largemotor.spin( speed, ~time~ )

Takes a `speed` in the range `[-100, 100]` (all values outside the range are
constrained) and runs the motor at that speed. Optionally, a `time` can be
passed in seconds and the motor will only spin for the allotted time before
stopping.

stop

    largemotor.stop()

Equivalent to `largemotor.spin( 0 )`.

\subsubsection{sensor}

distance

\subsubsection{smallmotor}

spin

stop

\subsection{CONSTANTS}

\subsubsection{BUTTON\_PIN}

\subsubsection{CYCLE}

\subsubsection{DC\_PIN}

\subsubsection{FRQ}

\subsubsection{GROVE\_PIN}

\subsubsection{SERVO\_PIN}

\subsection{FUNCTIONS}

\subsubsection{loop}

\subsubsection{pause}

    pause( ~time~ )

If a `time` is passed, waits for the allotted `time`. Otherwise, waits for
0.005 seconds.

\subsubsection{until}

    until( condition )

Pauses the program until the passed `condition` is satisfied.

\subsubsection{stopall}

    stopall()

Halts all constructed `large`/`smallmotors` (tracked in `dc\_motors` and
`servos`).

\subsection{VARIABLES}

\subsubsection{dc\_motors}

Tracks all constructed `largemotors`.

\subsubsection{servos}

Tracks all constructed `smallmotors`.

\end{document}
